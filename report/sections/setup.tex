\section{Setup}
\label{sec:setup}

\subsection{Hardware}
% TODO: Describe the robot we are designing for
% TODO: Describe the sensors available
% TODO: Describe that we assume the robot to have some sort of communication method

\subsection{Software Structure}
% TODO: Create a nice diagram
    % TODO: Clearly show where ROS is used (and not used)

\section{Software Structure}
The software is divided into three main parts, the robot behavior implementation, a simple custom 2D
simulator, and a ROS2 environment.

\subsection{Robot Behavior}
The robot behaviors are implemented in the botbrain rust crate. This crate defines the Robot interface
which is the only way for outside crates to interact with a robot’s internal state (see \cref{fig:robot-interface}).

\begin{figure}[H]
    \begin{center}
        \begin{minted}[autogobble]{rust}
            pub trait Robot {
                fn set_id(&mut self, id: RobotId);                 // Set the id of the robot
                fn set_params(&mut self, params: RobotParameters); // Set the parameters of the robot
                fn set_world_size(&mut self, size: Vec2);          // Set the size of the world

                fn input_pose(&mut self, pose: RobotPose);         // Input the angle of the robot
                fn input_cam(&mut self, cam: CamData);             // Input data from the camera
                fn input_lidar(&mut self, lidar: LidarData);       // Input data from the lidar
                fn input_msgs(&mut self, msgs: Vec<Message>);      // Input messages from other robots

                fn get_id(&self) -> &RobotId;                      // Get the id of the robot
                fn get_params(&self) -> &RobotParameters;          // Get the parameters of the robot
                ...
            }
        \end{minted}
    \end{center}
    \caption{Methods of the \texttt{Robot} interface.}\label{fig:robot-interface}
\end{figure}

Interface methods prefixed with \texttt{set} are used to set up the internal state of the robot and should be called after creating a robot and before any other methods. The \texttt{input} prefix is used in functions which should be called as often as possible to update the world view of the robot. \texttt{get}-methods do not change the state of the robot and may be called at any time. Notice that there is no method which "runs" the robot to generate a control signal. This is because a \texttt{Robot} only represents the robot state, not the algorithm which uses that state to generate a control output. To generate a control output, the dynamic \texttt{Robot} must be passed to a behavior function as defined in \cref{fig:behavior-fn}. This function is then free to read and write to the robot state, to generate a control signal as well as a list of messages which should be sent to the other robots.

\begin{figure}[H]
    \begin{center}
        \begin{minted}[autogobble]{rust}
            pub type BehaviorFn = fn(&mut Box<dyn Robot>, Duration) -> (Control, Vec<Message>);
        \end{minted}
    \end{center}
    \caption{Definition of a behavior function.}\label{fig:behavior-fn}
\end{figure}

Separating the robot state from its behavior allows multiple behaviors for the same internal state, which is useful for comparing different algorithms.

\subsection{Using Rust}

\subsection{Two Simulators}
% TODO: Why two simulators?
% TODO: Shared behavior
% TODO: Performance differences
