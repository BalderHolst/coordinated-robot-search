\section{Search Algorithms}
This section describes the development of a distributed algorithm which makes it possible for a group of robots to find a search item.

\subsection{Search Grid}
Each robot has a \texttt{search\_grid} which is continuously updated by observations by the robot itself, and the observations of other robots which it receives over its communication channel.
% TODO: Add a nice figure

\subsection{Proximity Grid}
% TODO: How is this populated
% TODO: Add a nice figure

\subsection{Target Contributions}
% TODO: Describe the idea, that different factors contribute to the target vector. The target vector is the sum over all contributions.
% TODO: Describe how we take gradients on a grid

\subsection{Global Planner}
% TODO: Why is if useful
% TODO: Show cost map and example of a planned route
