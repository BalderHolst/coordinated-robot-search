\section{Conclusion}
\label{sec:Conclusion}
This project presents a modular and extensible framework for coordinated robotic search, evaluated using both a lightweight 2D simulator and a high-fidelity 3D Gazebo environment. Several search strategies were implemented, including a simple obstacle-avoiding baseline, a gradient-based method, a hybrid gradient-pathing algorithm, and a Deep Q-Network (DQN)-based learning approach. The goal was to assess their performance in terms of coverage efficiency, computational cost, and cross-simulator consistency.

The experimental results provide support for several of the initial hypotheses:

\begin{itemize}
    \item The gradient-based algorithm significantly outperformed the baseline in terms of coverage efficiency, owing to its use of local map knowledge, confirming \textbf{Hypothesis 1}. However, as expected it introduced increased computational cost due to continuous gradient field computation.
    \item The hybrid approach achieved higher coverage efficiency than the gradient-based method, as expected, partially confirming \textbf{Hypothesis 2}. Interestingly, its average computational cost was lower than anticipated. This result challenges the original assumption but led to the insight that pure path planning may serve as a viable alternative to gradient-driven navigation in some scenarios.
    \item The comparison between \texttt{simple\_sim} and Gazebo showed strong consistency in coverage trends, even accounting for differences in physical modeling and sensor noise. This was achieved withThis supports \textbf{Hypothesis 3} and validates the use of the lightweight simulator for rapid development and evaluation.
    % \item The RL-based search showed promising results and demonstrated the potential to match the performance of manually engineered algorithms, partially supporting \textbf{Hypothesis 4}. However, it exhibited greater variability and required further tuning to achieve consistent performance.
\end{itemize}

Overall, the results highlight that distributed, modular behavior can be developed and tested efficiently in simulation, and later deployed to real-world systems with minimal modification. The framework’s hardware abstraction and ROS 2 integration further support its portability and extensibility.\\

\textbf{Future work} will focus on deploying the system on physical robot platforms, addressing real-world challenges such as communication constraints, sensor calibration, and robust localization. Enhancing the reinforcement learning component, particularly the DQN implementation, will also be a priority. Promising directions include exploring Proximal Policy Optimization (PPO), Neuroevolution of Augmenting Topologies (NEAT), and sim-to-real transfer methods to enable scalable and adaptive multi-robot behaviors.
