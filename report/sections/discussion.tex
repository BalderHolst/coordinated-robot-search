\section{Discussion}
\subsection{Hypothesis 1}
As seen in \cref{sec:search-algorithms-benchmark} the gradient-based algorithm outperforms the baseline obstacle-avoidance algorithm but at a higher cost, see \cref{fig:computation-performance}.
This hypothesis can be confirmed.


\subsection{Hypothesis 2}
As seen in \cref{sec:search-algorithms-benchmark} the hybrid algorithm outperforms the gradient-based algorithm but at a lower average cost, see \cref{}. It was not expected to have a lower avegarge cost than the purely gradient-based approach, as the hybrid algorithm combines both gradient-based navigation and high-level path planning.
This hypothesis can not be confirmed.

\subsection{Hypothesis 3}
In \cref{sec:simulator-consistency} the basic movement and the search-algorithms was compared to evaluate the if trends in the Gazebo simulator can be reproduced in \texttt{simple\_sim}. 

The coverage progression for all search algorithms fall within 15 \% of each other when running in Gazebo vs \texttt{simple\_sim}, see \cref{fig:coverage-benchmark-all}. 

From the basic movement calibration, see \cref{fig:movement-consistency}, the Gazebo path deviates slightly from a perfect circle. This deviation is attributed to realistic physical effects such as wheel slippage and friction between the wheels and the ground. In contrast, the \texttt{simple\_sim} simulator assumes ideal motion without any loss due to slippage or friction, resulting in a more precise circular trajectory. The same goes for the straight line movement, where the floor might not be perfectly flat, resulting in a slightly curved path.

The Roomba algorithm is however faster in Gazebo. This could be due to a larger number of lidar rays in the Gazebo simulation giving the robot more stable information about the closest obstacle to the robot. 

It shows that the simple simulator performance is generally similar to Gazebo, but can not be relied upon as a single source of performance benchmarks, as some behaviors perform dispropportionally better or worse in Gazebo as compared to \texttt{simple\_sim}. Especially the performance of the Gradient, Pure Pathing, and Hybrid algorithm seem to be be comparable across the simulators. 

With this in mind, the \texttt{simple\_sim} simulator run upwards of XX times faster than Gazebo, see \cref{fig:simulator-performance}, making the slight differences worth it.

It can be concluded that the lightweight simulator can approximate the key performance characteristics of Gazebo, while offering significantly reduced computational requirements. 

\subsection{Hypothesis 4}
Not relevant A.T.M.
