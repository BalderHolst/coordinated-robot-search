\section{Introduction}
\label{sec:Introduction}

% Where can it be used?
%  - Search and rescue
%  - etc.
% Same princip for different kind of missions

% Discuss the report structure

Coordinated robotic search presents a powerful solution to a variety of challenges.
Applications include search and rescue in disaster zones, environmental monitoring, industrial inspection, and surveillance. 
Multi-robot systems offer notable advantages in these contexts, such as increased coverage efficiency, robustness through redundancy, and the ability to operate in environments that may be hazardous or inaccessible to humans.\\

Despite differences in specific objectives, many robotic missions are governed by shared principles. 
Robots must autonomously explore known or partially known environments, interpret sensor data to detect targets or obstacles, and maintain coordinated behavior while constrained by limited communication ranges. 
This work focuses on the development and evaluation of distributed algorithms tailored to distributed algorithms for searching a mapped aera, using simulated environments to assess performance.\\

To support development and evaluation, a dual-simulator approach is employed: a lightweight 2D simulator for rapid iteration and visual debugging, and a more realistic Gazebo-based 3D simulation integrated with ROS 2.
These environments allow for testing under realistic sensing and communication constraints.\\

The following sections detail the problem statement, review related work, describe the hardware and software setup, and explain the structure of both simulators. 
Subsequently, the developed search algorithms are presented, followed by an exploration of a deep reinforcement learning-based alternative, and concluding with an evaluation and discussion of future work.
