\section{Real World Implementation}
While this project has not been deployed on a physical robot, the entire software stack has been developed with real-world implementation in mind. 
Both the simulation architecture and behavioral algorithms are designed to be modular, portable, and compatible with commonly available robotics hardware and middleware. 
This section outlines how the system could be transferred from simulation to real-world execution.

\subsection{Using ROS 2}
The primary method for deploying the behavior algorithms is through a dedicated ROS 2 node included in the project. 
This node interfaces with the robot’s onboard sensors and actuators using standard ROS 2 topics. 
Specifically, it requires subscribtion to the following input topics:

\begin{itemize}
  \item \texttt{/scan}: LiDAR data, used for obstacle detection and localization.
  \item \texttt{/camera/image}: Image stream from the onboard RGB camera, used for object detection.
  \item \texttt{/amcl\_pose}: Robot pose estimated via AMCL (Adaptive Monte Carlo Localization), providing global localization within a known map.
\end{itemize}

These topics are typically published by standard ROS 2 packages, such as \texttt{nav2}, and drivers for commonly used platforms like the TurtleBot 4. 
The behavior node publishes control commands to the \texttt{/cmd\_vel} topic, which specifies linear and angular velocities for the robot.

In simulation, the differential drive system is modeled within Gazebo using its built-in plugins. 
However, as Gazebo is not intended for real-world deployment, a separate controller is required for physical robots. 
A widely adopted solution is the \texttt{ros2\_control} framework \cite{ros2-control}, which offers a collection of hardware interface controllers, including one for differential drive systems.
This framework also includes a Gazebo plugin that allows the same controller to be used both in simulation and on real hardware, ensuring consistency across development and deployment environments.

\subsection{Using the Rust Library}
For platforms or embedded systems that do not support ROS 2, the behavior algorithms can be deployed using the standalone \texttt{botbrain} Rust library. 
This library exposes a high-level \texttt{Robot} interface which accepts sensor data and returns control commands and outgoing messages for communication.

% TODO: Maybe a simple code example??

This approach allows for flexible integration in environments where ROS 2 is not viable, such as microcontroller-based systems or custom embedded platforms.

\subsection{Communication Considerations}\label{sub:communication}
In simulation, message passing between robots is instantaneous and lossless. 
In a real-world implementation, communication can become a limiting factor. 
Several technologies could be considered:

\begin{itemize}
  \item \textbf{Wi-Fi}: High bandwidth, but may suffer from interference or limited range in large-scale outdoor applications.
  \item \textbf{Zigbee}: Low power and suitable for mesh networks but has limited bandwidth and range. Might not support high-frequency updates or large message payloads.
  \item \textbf{LoRa}: Long-range communication with minimal bandwidth. Only suitable for infrequent or compact messaging.
\end{itemize}

The system assumes one shared communication channel, where all robots broadcast their observations. 
Message serialization is already implemented in a compact binary format to reduce communication overhead. 
If communication becomes a bottleneck, strategies such as message prioritization or selective transmission (e.g., only sending certain changes) may be necessary.

\subsection{Hardware Requirements}
The project targets a differential drive robot equipped with:
\begin{itemize}
  \item A 2D LiDAR scanner.
  \item An RGB camera with a known field of view.
  \item Wheel odometry or IMU for dead reckoning.
  \item A communication module (e.g., Wi-Fi or Zigbee).
\end{itemize}

While the TurtleBot 4 is used in simulation, the system can be adapted to other platforms by updating sensor interfaces and adjusting configuration parameters.
