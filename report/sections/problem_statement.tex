\section{Problem Statement}
\label{sec:problem-statement}

This project focuses on developing distributed algorithms that enable collaborative robot search for a predefined object within a predefined and mapped environment. Both manually developed and deep reinforcement learning approaches are explores. During a search mission, the environment is progressively explored until the target object is located. The effectiveness of each search algorithm is evaluated using three parameters:

\begin{itemize}
    \item Map coverage over time
    \item The computational expense of the algorithm
\end{itemize}

All software developed in this project is tested exclusively in simulation; therefore, the specific communication technology is not defined. Robots are, however, constrained by a limited communication range, and the behavior algorithms should ensure that robots remain within this range to emulate real-world communication constraints. Each robot is equipped with a camera for detecting the search object and a LiDAR sensor for measuring distances to objects. \\

To evaluate the algorithms and demonstrate their feasibility and support easy integration for ROS 2 based systems, they are implemented as ROS 2 nodes for simulation in Gazebo. \\

The project also includes the development of a simple simulation environment, which facilitates the development of search algorithms by providing visual debugging capabilities without depending on ROS 2 stack, enabling a faster iteration cycle. This environment also facilitates training agents using reinforcement learning. \\

Four concrete hypotheses are formulated to guide development and evaluation of the project:

\begin{itemize}
    \item \textbf{Hypothesis 1:} A gradient-based search algorithm will outperform a basic obstacle-avoidance (baseline) algorithm in terms of coverage efficiency in a known map, but will incur higher computational cost.
    \item \textbf{Hypothesis 2:} A hybrid algorithm that combines gradient-based navigation with high-level path planning will achieve better coverage efficiency than both the baseline and purely gradient-based methods in a knwon map, but will incur higher computational cost.
    \item \textbf{Hypothesis 3:} A lightweight, custom-built simulator can approximate the key performance characteristics (e.g., coverage progression) observed in a high-fidelity simulator like Gazebo, while significantly reducing computational requirements.
    \item {\color{red}\textbf{Hypothesis 4:} A reinforcement learning-based search algorithm can achieve coverage efficiency comparable to that of gradient-based and hybrid approaches in a known map.}
\end{itemize}
