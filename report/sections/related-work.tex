% TODO: Make subsections explaining each type or delete some
\section{Related Work}
Multi-Robot Systems (MRS) represent a well-established field within robotics. MRS architectures are generally categorized as either centralized or distributed. In centralized approaches, a central server is responsible for assigning tasks and planning paths for all robots in the system. 
In contrast, distributed algorithms enable each robot to perform local planning and decision-making, often based on partial knowledge and limited communication. \\

Notable approaches to MRS control include market-based methods~\cite{trigui2014market}, dynamic programming~\cite{kato2011dp}, behavior-based strategies, and deep reinforcement learning~\cite{huttenrauch2019deep-swarm}. In market-based task assignment, robots participate in an auction by bidding on tasks, with the highest bidder receiving the assignment. This method can lead to near-optimal task distributions, especially when task costs are well-defined. Dynamic programming approaches aim to compute optimal task allocations and trajectories based on known locations and capabilities of all robots. These methods typically require complete, up-to-date global information and are thus most suitable for centralized systems, or for distributed systems in which the full system state is shared \cite{multi-robot-search-moving-target}. Behavior-based approaches often draw inspiration from biological systems, such as ants and bees, and use indirect coordination methods like virtual “pheromones” or other spatial markers to guide robot behavior. In \textit{Cooperative transport by ants and robots} C. R. Kube and E. Bonabeau found that “a coordinated group effort is possible without use of direct communication or robot differentiation” \cite{kube2000cooperative-ants}. Such strategies reduce the reliance on explicit communication and instead emphasize emergent, self-organizing behaviors. \\

This project adopts a similar approach by enabling robots to share a continuous stream of observations over a communication channel. These collective observations are used to update each robot’s internal representation of the search map, which in turn influences its navigation decisions. This form of implicit communication allows for scalable coordination, as the time complexity of the algorithm does not increase with the number of participating robots \cite{multi-robot-search-moving-target}. \\

Three algorithmic approaches are explored and implemented in this work: a gradient-based method, a frontier exploration algorithm using A* for path planning, and a hybrid strategy combining the two. A deep reinforcement learning approach is also explored, but not fully realized. While these methods may not always produce optimal exploration strategies, they are designed to be robust, computationally efficient, and adaptable to a variety of real-world scenarios.
